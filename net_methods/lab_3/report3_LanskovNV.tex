%%%
% set up document type
%%%
\documentclass[12pt]{article}

%%%
% declare all packages
%%%
\usepackage[left=25mm, top=20mm, right=25mm, bottom=30mm,nohead,nofoot]{geometry} 

\usepackage[T2A]{fontenc}
\usepackage[utf8]{inputenc}
\usepackage[english, russian]{babel}

\usepackage{graphics, graphicx}

\usepackage{url}
\usepackage{hyperref}

\usepackage{amssymb,latexsym} 
\usepackage{MnSymbol}
\usepackage{mathrsfs}

\usepackage[nottoc,numbib]{tocbibind}
\usepackage{float}
\usepackage{listings}
\usepackage{multirow}
\usepackage{hhline}
\usepackage{delarray}

\usepackage{color,colortbl}

% \usepackage{verbatim}
%%%
% document settings
%%%
\setcounter{tocdepth}{4}
\graphicspath{ {./pic/} }

\renewcommand{\listoffigures}{\begingroup  % add number to list of graphics
\tocsection
\tocfile{\listfigurename}{lof}
\endgroup}
\renewcommand{\listoftables}{\begingroup  % add number to list of tables
\tocsection
\tocfile{\listtablename}{lot}
\endgroup}

%******************************************************************
%******************************************************************
\begin{document}

\begin{titlepage}
	\center
		Санкт-Петербургский Политехнический 
		университет \\ Петра Великого\\
		Институт прикладной математики и механики
		\\ \textbf{Высшая школа прикладной математики и вычислительной физики}

	\vfill ~
	\textbf{
		\\ \large ЛАБОРАТОРНАЯ РАБОТА №3
	}
	\\	на тему 
	\\ "Метод конечных объёмов для уравнений эллиптического типа"
	\\ по дисциплине
	\\ "Конечно-разностные и сеточные методы"

	\vfill ~

	Выполнил студент гр. \textbf{3630102/60101} \\
	\textbf{Лансков.Н.В.} \\ 

\vfill

{\large}	Санкт-Петербург
\\ 2019
\end{titlepage}

%%%
% Table of conetnts 
%%%

\tableofcontents 
\newpage
\listoffigures
\newpage
\listoftables
\newpage

%%%
% Text
%%%
\section{Постановка задачи}
\section{Метод конечных объёмов}

Рассмотрим  процесс нахождения коэффициентов по методу конечных объёмов.

Разобъём нашу область (прямоугольник) на конечные объёмы (в узлах). Тогда для каждой внутреней точки рассматриваем конечный объём $\Omega_{ij}$. Далее приведём вычисления в общем виде для такого конечного объёма.

Проинтегрируем уравнение 1 по конечному объёму и домножим на -1:
$$
\int\limits_{\Omega_{ij}}{\left[ \dfrac{\partial}{\partial x}\left( a\dfrac{\partial u}{\partial x}\right) + \dfrac{\partial}{\partial y}\left( b\dfrac{\partial u}{\partial y}\right) \right]}
$$

\section{Метод Якоби}
\section{Метод Зейделя}
\section{Метод SOR}
\section{Результаты}
\subsection{Метод Якоби}
\subsection{Метод Зейделя}
\subsection{Метод SOR}
\subsection{Сравнение методов}
\section{Выводы}
\section{Приложения}
Исходные файлы лабораторной работы можно найти тут: \\
\url{a}
\end{document}

