%%%
% set up document type
%%%
\documentclass[12pt]{article}

%%%
% declare all packages
%%%
\usepackage[left=25mm, top=20mm, right=25mm, bottom=30mm,nohead,nofoot]{geometry} 

\usepackage[T2A]{fontenc}
\usepackage[utf8]{inputenc}
\usepackage[english, russian]{babel}

\usepackage{graphics, graphicx}

\usepackage{url}
\usepackage{hyperref}

\usepackage{amssymb,latexsym} 
\usepackage{MnSymbol}
\usepackage{mathrsfs}

\usepackage[nottoc,numbib]{tocbibind}
\usepackage{float}
\usepackage{listings}
\usepackage{multirow}
\usepackage{hhline}
\usepackage{delarray}

\usepackage{color,colortbl}

% \usepackage{verbatim}
%%%
% document settings
%%%
\setcounter{tocdepth}{4}
\graphicspath{ {./pic/} }

\renewcommand{\listoffigures}{\begingroup  % add number to list of graphics
\tocsection
\tocfile{\listfigurename}{lof}
\endgroup}
\renewcommand{\listoftables}{\begingroup  % add number to list of tables
\tocsection
\tocfile{\listtablename}{lot}
\endgroup}

%******************************************************************
%******************************************************************
\begin{document}

\section{Постановка задачи}

\quad  Дано:
\begin{enumerate}
\item Полигональная область $\Omega$ с треугольной сеткой
\item Заданная функция (например, u(x, y) = sin(x + y))
\end{enumerate} 

Задача:

\begin{enumerate}
\item Построить линейный интерполянт $I_h$
\item Написать функцию, вычисляющую норму $L^2(\Omega)$ разности точного решения и интерполянта
\item Написать функцию, вычисляющую норму $L^2(\Omega)$ разности градиентов точного решения и интерполянта 
\item Найти зависимость точности вычисления норм от шага $h$
\item Написать алгоритм минимизации функционала вида:

\begin{equation}
J = ||u_h - u||_{L^2}^2
\end{equation}

\end{enumerate} 




























\end{document}