%%%
% set up document type
%%%
\documentclass[12pt]{article}

%%%
% declare all packages
%%%
\usepackage[left=25mm, top=20mm, right=25mm, bottom=30mm,nohead,nofoot]{geometry} 

\usepackage[T2A]{fontenc}
\usepackage[utf8]{inputenc}
\usepackage[english, russian]{babel}

\usepackage{graphics, graphicx}

\usepackage{url}
\usepackage{hyperref}

\usepackage{amssymb,latexsym} 
\usepackage{MnSymbol}
\usepackage{mathrsfs}

\usepackage[nottoc,numbib]{tocbibind}
\usepackage{float}
\usepackage{listings}
\usepackage{multirow}
\usepackage{hhline}

\usepackage{color,colortbl}

% \usepackage{verbatim}
%%%
% document settings
%%%
\setcounter{tocdepth}{4}
\graphicspath{ {./pic/} }

\renewcommand{\listoffigures}{\begingroup  % add number to list of graphics
\tocsection
\tocfile{\listfigurename}{lof}
\endgroup}
\renewcommand{\listoftables}{\begingroup  % add number to list of tables
\tocsection
\tocfile{\listtablename}{lot}
\endgroup}

%******************************************************************
%******************************************************************
\begin{document}

\begin{titlepage}
	\center
		Санкт-Петербургский Политехнический 
		университет \\ Петра Великого\\
		Институт прикладной математики и механики
		\\ \textbf{Высшая школа прикладной математики и вычислительной физики}

	\vfill ~
	\textbf{
		\\ \large ЛАБОРАТОРНАЯ РАБОТА №2
	}
	\\	на тему 
	\\ "Исследование разностных схем для параболических уравнений"
	\\ по дисциплине
	\\ "Конечно-разностные и сеточные методы"

	\vfill ~

	Выполнил студент гр. \textbf{3630102/60101} \\
	\textbf{Лансков.Н.В.} \\ 

\vfill

{\large}	Санкт-Петербург
\\ 2019
\end{titlepage}

%%%
% Table of conetnts 
%%%

\tableofcontents 
\newpage
\listoffigures
\newpage
\listoftables
\newpage

%%%
% Text
%%%
\section{Постановка задачи}

Рассмотрим задачу :

$$
\begin{cases}
\dfrac{du}{dt} - \rho(x)\dfrac{d^2u}{dx^2} = f(x, t), & x \in [0.5;2],  t \in [0, T] \\ \\
0 < \rho_{min} < \rho(x) < \rho_{max}, & \rho(x) = (x+1)*10^{-7} \\ \\
 - \dfrac{du}{dx}(0.5) + u(0.5) = \mu_1(t) \\ \\
\dfrac{du}{dx}(2) + u(2) = \mu_2(t)
\end{cases}
$$

Где:
$$
\begin{cases}
f(x, t) = -3sin(e^x)xe^{-3t} - x(2e^{-3t+x}cos(e^x)+e^x(cos(e^x) - e^xsin(e^x)))  \\  \\ 
\mu_1(t) = -(e^{-3t}sin(e^{0.5}) + e^{0.5}cos(e^{0.5})(0.5e^{-3t}+1))+sin(e^{0.5}(0.5e^{-3t}+1)))\\ \\
\mu_2(t) = (e^{-3t}sin(e^{2}) + e^{2}cos(e^{2})(2e^{-3t}+1))+sin(e^{2}(2e^{-3t}+1)))
\end{cases}
$$

Решение ищем в виде : 
$$
\begin{cases}
u^*(x) = v(x, t)w(x) \\ \\
w(x) = sin(e^x) \\ \\
v(x, t) = x*e^{-3t}+1
\end{cases}
$$ 
\section{Разностные схемы}
\subsection{Явная схема}

\subsection{Неявная схема}

\subsection{Симметричная схема}
\section{Результаты}

\section{Выводы}

\end{document}

